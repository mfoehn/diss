% !TEX root = arbeit.tex
\section{Conclusion}
	% Write the detector part.	
	
	The thesis consists of two main parts. Chap.~\ref{sec:theory} showed theoretical analyses of different components of the NIM instrument and summed up theoretical aspects to understand the performance results of the different subsystems presented in Chap.~\ref{sec:Exp}. Chap.~\ref{sec:Exp} ends with test results of the NIM PFM and FS sensors. This chapter sums up the most important findings from the calculations and tests from the before mentioned chapters.\\

	% Pulse generator simulations and flight pulse generator performance. Summarize. ok
	The first analysis that was performed was a simulation to compare the impact of the ion temperature, the spacial spread of the ions in the ionisation region and the impact of the fall time of the high-voltage extraction pulse on the mass resolution (Chap.~\ref{chap:massRes}). The simulations revealed that compared to the other two effects the impact of the temperature on the mass resolution is negligible. The fall time of the extraction pulse has a major impact on low mass species. The fall time changes the potential in the ionisation region during the time when the ions get extracted. Ions with a lower mass/charge ratio (m/z) are therefore more affected. For a fall time of 5~ns, corresponding to the fall time of the flight extraction pulse, an ion with m/z of 1 has a 5~\% lower mass resolution than an ion with m/z 200. When the fall time is smaller than 1~ns, the impact on the mass resolution is below 0.1~\% according to the simulations. This fall time should be the target time for future pulse generator designs.
	The spacial spread of the ions in the ionisation region has the biggest impact on the mass resolution. To minimize it, proper focusing of the ions in the ionisation region is necessary of the ion-optical lenses.\\
	% Leave that part away because no detailed analysis about the ringing has been performed. Discuss it with Peter.
	The flight pulse generator fulfils most of the requirements. For future designs there is potential in reducing the ringing of the baseline voltage. The baseline voltage is the low voltage applied on the extraction grid during the time between two extraction pulses and is one of the voltages generating the electric field in the ionisation region to trap the generated ions (Chap.~\ref{chap:ExpPulser}). The ion storage behaviour depends heavily on the stability of the voltages in the ions ionisation region. Already a fluctuation of 0.1~V can lead to a major loss of ions because they leak out of the source.\\ % More explanations about the different requirements to the pulse shape.
	% Density enhancement. Proofread the part from the entrances holes on. Ok except marked part
	A detailed analysis of the different factors influencing the density enhancement ability of a closed source antechamber has been performed in Chap.~\ref{subsubsec:Densenhan}. In this summary I focus on the parameters influencing the geometry of the antechamber. For details about the impact of the spacecraft velocity or the particle mass, read Chap.~\ref{subsubsec:Densenhan}. The analysis showed, that the diameter of the hole connecting the antechamber with the ionisation region should to be bigger than 5~mm with the geometry of NIM's antechamber. A big hole area increases the probability that the thermalised particles from the antechamber flow into the ionisation region and not out of the entrance holes. In addition, the conductance of a tube increases with increasing tube cross-section area compared to the tube length \cite{molFlowTubeTransm_Essen1976}. The tube length is given because it has to have a certain length to build in a shutter to close this entrance when measurements with the open source channel are conducted. Therefore, the tube area cross-section should be larger compared to the tube length in future designs of such antechambers.
	
	% Motor inbetween
	The shutter to close the entrance between the antechamber and the ionisation region is a very important part in the NIM design because when measuring with the open source channel, the aim is to measure neutral particles and ions directly meaning that they should not interact with any surface of the spacecraft. Neutral particles entering through the closed source antechamber will hit multiple times the antechamber's inner walls to get thermalised and have the opportunity to interact with the wall material. With the open source channel, neutral particles and ions enter the ionisation region undisturbed. Therefore it is important to close the entrance to the antechamber properly to guarantee that as less particles as possible enter the ionisation region from the antechamber when measuring with the neutral or the ion mode channel. The shutter is a plate with an opening in the plate. The plate lies in a pocket between the antechamber and the ionisation region. When the shutter is closed, the plate moves to the side and the gas has to flow around the shutter to reach the ionisation region through the antechamber. The amount of gas reaching the ionisation region through this path depends heavily on the gap size between the shutter plate, the pocket and the antechamber. Simulations revealed that with decreasing gap size, the attenuation increases rapidly. A gap size of 0.01~mm results in an attenuation of a factor 600 where a gap size of 0.1~mm results in an attenuation of 25 (Chap.~\ref{subsubsec:motorflow}).\\
	
	The particle reflection coefficient of the coating of the antechamber inner walls has to be very close to 1. Otherwise the particles get absorber by the chamber walls instead of reflected. For the geometry of the NIM antechamber a particle reflection coefficient of 0.999 instead of 1 results in a signal reduction of ~30~\%.\\
	
	
	The impact of the density enhancement behaviour is also visible when measuring with the PFM entrance slit (Chap.~\ref{chap:expDensEntSlit}). When the neutral particle beam is directed on the filament bloc or the metal sheet opposite of the filament bloc, the neutral particles get decelerated leading to an enhanced density in the ionisation region. The purpose of the open source channel is to measure incoming neutral particles and ions directly that they do not interact with the structure of the instrument \notes{Rewrite}. An analysis of the data conducted with the NIM Prototype revealed that a pillar instead of a metal sheet would have been a better choice as a supporting structure because incoming particles get deflected in all directions instead of being channeled towards the central extraction grid.\\
	% May mention here also the improvement of the antechamber
	% Is the position of the entrance holes and the resulting plateau such relevant to mention it in the conclusions? & -> ILENA Meeting
	% FoV Analysis: ok
	The FoV analysis of Chap.~\ref{subsubsec:Calfly} revealed, that NIM's FoV is blocked for angles bigger than 100° from the spacecraft. Simulations showed that during the flybys at the icy moons' NIM has to change very rapidly between thermal and neutral mode. Due to the entrance hole positions of the closed source antechamber, the antechamber is still able to measure also in the FoV of the neutral gas channel although with a reduced signal intensity due to the bad inflow angle for the antechamber. Therefore it is very important to switch between the two modes as fast as possible because during the switch-over time NIM cannot record any spectra resulting in a loose of precious data.\\
	% Motor. Rewrtie mainly the start and fusion it with the part above about the density enhancement of the entrance slit.
	
	% Detector. Mechanical part still not so well :/.
	The mechanical and electrical design of the detector was improved (Chap.~\ref{sec:setup}). In the mechanical design, the detector housing was improved to easier assemble the MCPs and the electrical contact ot the contact lug to the MCPs was also improved. In the electrical design the Zener diode was replaced by a resistor because the resistor is more robust concerning discharges. It is a step back in the design but because the detector suffered frequently discharges, it was a necessary solution to improve the robustness of the circuitry with regards to discharges.\\
	
	% Sensor tests
	Finally, the NIM PFM and FS sensors were tested. The sensors were operated with laboratory electronics because in the case of the PFM, there was only very little time to test the whole system before it was delivered to the spacecraft in December 2020 and in the case of the FS instrument, the complete electronics was not ready at the time writing this thesis.\\
	Both sensors showed good performance regarding mass resolution and SNR. The FS sensor shows with 830 for thermal mode and 706 for neutral mode a slightly better performance regarding mass resolution than the PFM, which has a mass resolution of 757 for thermal mode and 534 for neutral mode respectively.	Both sensors showed a SNR of almost 6 decades which is within the requirements of the two sensors. The NIM PFM and FS ion-optical systems showed both a very good ion storage behaviour. The reason lies in the special designed ion storage source consisting of ring electrodes to store the electrons in 2 directions where the ionising electron beam traps the generated ions in the third direction generating a potential trap for the ions.\\
	
	In conclusion, the NIM PFM and IS sensor were successfully tested. Different subcomponents were qualified and the analysis showed different drawback which future designers of such instruments may should take care of a bit more than we did.
	
	
	
	\begin{comment}
		This thesis follows up the PhD thesis of Stefan Meyer \cite{Diss_Meyer}. At the end of his 	thesis, the NIM prototype was built and the design of the NIM Proto~Flight~Model (PFM) was almost completed.\\	
		The objective of this thesis was to finalize the design of the NIM flight model, to build and 	test the NIM PFM to deliver it to the JUICE spacecraft and to build and test the NIM Flight-Spare (FS) model, which stays on Earth as a ground reference model. This required environmental tests of various flight subcomponents as they got available to verify their functionality. It required ion-optical simulations to set constrains on the design of the flight power supplies as they were still under development during the early phase of this thesis. In the later phases of the dissertation, tests with the whole system (ion-optical system operated with flight electronics and software) were performed to support the software development.\\
		The thesis consists of three main parts: Chap.~\ref{sec:theory} shows the theoretical 	performance of some key components of the NIM instrument such as the performance of the closed source antechamber or the ion storage behaviour of NIM's ionisation region. Chap.~\ref{sec:setup} compares the design of the NIM prototype with the final flight design and shows the main differences between the two models. Chap.~\ref{sec:Exp} shows test results of flight components tested stand alone, such as the detector, or attached to the NIM prototype, such as the closed source antechamber or the ion-mirror. It ends with performance results of the NIM PFM and FS ion-optical systems.
	\end{comment}

	\begin{comment}
	
		\begin{itemize}
			\item I did a theoretical simulation of the high-voltage pulse generator and showed the 	impact of different effects influencing the mass resolution. In addition, I tested the flight pulse generator boards and showed their performance. % and what are the key results? That acoring to laboratory expertise ionstorage with the flight power supply is not possible? Or just more difficult?
			\item I did a detailed analysis of the performance of the closed source antechamber and show the impact of the different parameters. I verified the theory and showed that in our design also the open source entrance slit shows similar behaviour as the antechamber due to a bad design.
			\item I showed simulations of a sample flyby of the JUICE satellite on Callisto to show the changing gas inflow direction and to show, how important it is to change fast between the thermal and neutral measuring mode.
			\item I showed calculations of the shutter performance used to close the entrance between 	the closed source antechamber and the ionisation region. The measurements used to verify the shutter performance revealed that the shutter build in the PFM instrument has tolerances of 0.1~mm resulting in an attenuation factor of only 25 instead of the optimal 600 when the shutter is properly fabricated.
			\item I showed the design improvements with regards to the mechanical and electrical design 	on the NIM detector with the output, that for future missions, the mechanical design still lacks on robustness.
			\item I showed performance of the NIM PFM and FS sensor. The FS showed with a mass 	resolution of 830 a slightly better performance than the PFM. With respect to SNR both sensors showed similar performance. Both sensors showed a very good ion storage behaviour compared to the prototype. The PFM has a verified mass range of 642~u.
		\end{itemize}
		In summary, the main focus in theoretical part was of on the investigation of the performance of the closed source antechamber. In the experimental part, the focus was on the performance tests of the two complete sensors.
	
	\end{comment}


	% Wichtigste Punkte nocheinmal zusammenfassen (Prototyp tests und PFM/ FS tests)
	% Put the final version of the PFM/ FS instrument into this chapter with all the components, Antechamber version, IS... as a summary of the instrument at its final state as it is at the end of the thesis. Just as a short state description.
	
	% Theoretical analysis of the antechamber and motor performance. showing the impact of the high voltage pulser fall time with a simplified model.
	% Test of different sub components of the NIM instrument such as antechamber and Motor
	
	\begin{comment}
		The NIM PFM was successfully tested, delivered and integrated onto the JUICE spacecraft in December 2020. The NIM FS sensor waits now until the JUICE spacecraft has started its journey in September 2022 to Jupiter. This chapter summarizes the key results of the different calculations, simulations (Chap.~\ref{sec:theory}) and performance tests of the subcomponents and the complete NIM PFM and FS ion-optical system (Chap.~\ref{sec:Exp}).\\
	\end{comment}
	
% !TEX root = arbeit.tex
\section{Conclusion}

	During this thesis, the flight design of the NIM instrument was finalised and the NIM PFM and FS instruments were built and tested.
	
	The design and building of such an instrument depends on a lot of different factors. The scientists ask a specific question and according to that question they define the instrument and set constraints on the instrument performance.
	% Some of these constraints are looked at in a bit greater detail once an example looking a bit at the operating electronics and one example concerning constraints originating from the mission.
	On two examples I show the worked performed during the thesis and which impact these constraints had on the design of the instrument to get the final flight instrument.\\
	
	% Pulser and flyby examples
	With regards to NIM, some of the key performance parameters are the mass resolution and the signal-to-noise ratio to have the sensitivity to determine the species and densities to a sufficient accuracy in the icy moons' exospheres. To achieve such a high performance, the operating electronics is pushed to its limits. According to the high-voltage pulse generator used to extract the ions from the ionisation region, a few examples of such requirements are presented.\\
	Two of the most important parameters concerning that specific element are the fall time and the bias voltage. The fall time has to be as short as possible to ensure that all ions get the same amount of energy. Otherwise, the mass resolution of especially low mass ions suffers (see Chap.~\ref{chap:massRes} for the theoretical analysis). Typical fall times are in the range of a few ns. Two different flight designs were tested during the thesis which showed a clear preference to the one with the shorter fall time according to the results \cite{Lasi_IEEE2020}. Another important parameter is the bias voltage is the voltage applied on the extraction grid, when no high voltage pulse is applied. As shown in Chap.~\ref{chap:FSCalib} and \cite{Foehn2021}, the PFM and FS have both a good ion storage capability. To achieve that it is necessary to have a stable potential well in the ionisation region during the time when no extraction pulse is applied on the extraction grid. This requires that the voltages applied on the electrodes in the ionisation region due not fluctuate more than 100~mV. Laboratory tests showed that the most critical ones with respect to ion storage are the bias voltage applied on the extraction grid and the voltage applied on the grid opposite of the extraction grid. The flight pulse generator shows lees good performance with regards to that parameter (see Chap.~\ref{chap:ExpPulser}) but with the other limitations such as derating or the radiation hardness of the components, it was the best possible solution.\\
	Other requirements appear from the trajectory of the spacecraft. JUICE contains several different instruments ranging from cameras, to gravity meters to particle detectors. The orientation of the spacecraft depends on the instruments to enable them optimal FoV on their target. For a spacecraft having solar panels for the power generation, it is important to be oriented perpendicular to the Sun whenever possible to optimize power generation especially for missions with targets such far away from the Sun as the mission JUICE with target Jupiter. For the NIM instrument in particular that meant to have a design with a big FoV. NIM has an open-source and a closed-source entrance to measure neutral particles and ions directly and to thermalise neutral particles with the closed source antechamber to make advantage of the density enhancement behaviour to amplify the signal with that mode. Simulations of the trajectory of the spacecraft revealed that two entrance holes on the antechamber were necessary to be able to measure during the different flybys although a second entrance hole leads to a lower performance of the closed-source antechamber (Chap.~\ref{subsubsec:Densenhan} and \ref{subsubsec:Calfly}).\\
	
	With all this, the design was finalized and the two flight models were built and tested. The NIM PFM was delivered in December 2020 to the JUICE spacecraft where it waits to start its journey to Jupiter and its icy moons. 
	
	
	
	
	
	
	
	
	
	
	
	
	
	
	
	
	
	
	
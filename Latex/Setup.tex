% !TEX root = arbeit.tex
\section{Setup}\label{sec:setup}
	\begin{comment}
	The goal of this chapter is to show NIM in the current lab equipment. An 'as is' state.
	Instrument in vacuum chamber. Mechanical description. Theoretical description is in the theory part.
	IS based on the design of Abplanalp 2009 if so. Look it up for NIM.
	With which IS, reflectron, detector. Short description of the different parts as they were for the first tests. (IS, reflectron, detector type). With pictures.
	Lab electronics, cabling? Or just reference to Stefan's Diss? Used standard settings such a pulser timings, UMCP, filament emission current, chamber pressure.
	(will also be mentioned in chamber experiments for the different settings)
	
	\end{comment}
	
	\subsection{NIM Instrument}\label{subsec:setupInst}
		\subsubsection{Prototype}
		% NIM Instrument description overview. Startconfiguration.
		\begin{comment}
		IS with 8 lenses and one filament holder (flight 7 lenses, 2 filament holders)
		small antechamber with edge, one entrance hole at 90° (fight 80mm diam., 2 holes at +-60° relative to normal)
		Prototype reflectron with ring electrodes (flight has 3 electrodes with a linear increasing resistance in between them)
		Pulser: lab pulser (flight still under development -> tests later)
		\end{comment}
		
		\subsubsection{Protoflight Model}
		% PFM Simulationen Stand. Instrumentenvergleich. Subchapter of NIM instrument
	
	\subsection{Test facilities/ Test Tools}\label{subsec:setupTestTools} % Bei test tools wäre das simulationsprogramm auch inbegriffen
	
	% Testanlagen hier beschreiben in einem einzelnen Unterkapitel falls sie weiter ausgeführt werden. Evt. auf CASYMIR nur einen Verweis machen. SATANS etwas genauer auseinander nehmen.
	% Explain the simulation programm as far as it is needed. This chapter will be referenced from the experiments chapter. In experiments, simulations and tests will be compared -> reference on how the simulations work are explained here as far as needed.
	% SATANS
	


%-------------------------------------------------------------------------------------	
	%% STROFIO chamber
	\begin{comment}
		Include here the whole cabling? No. Refer to Diss Stefan for the description of the start electronics. -> show the changes later in Chap. Experiments.
		Picture inside the STROFIO chamber. One picture of the current test setup with the rotation mechanism. Show on a picture where the gas enters the IS and the antechamber.
		Static gas measurement: Measurement with the STROFIO chamber with a set background gas. (Or also describe it in the different experiments? Make also a note there because then it is clear how the different measurements where performed)
		
		If there is not enough txt for a chapter, include it at the description of the NIM prototype.
	\end{comment}
	
	
%---------------------------------------------------------------------------------------
	%% Simulation Prototyp (in the simulationen, the simplification of the ionsource is already implemented)
	\begin{comment}
		7 lenses, no filament (you set the place and flight direction of the entering produced ions -> no filament in the current simulation model)
		
		flight reflectron
		
		Pulser: you can set the most important parameters of the pulse shape in the simulation.
		
		
		
		
		
		
		
		
		
	\end{comment}
	
	
%---------------------------------------------------------------------------------------
	% SATANS
	\begin{comment}
		further steps. Further investigation of the ion source to get more ions.
		
		
		
	\end{comment}
	
	
	
	

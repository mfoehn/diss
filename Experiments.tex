% !TEX root = arbeit.tex
\section{Experiments} \label{sec:Exp}
	The tests described in this chapter include tests of the different components of the NIM instrument such as laboratory tests of the flight antechamber tested on the Prototype or the flight ion-mirror. The part of Chapter \ref{sec:Exp} includes manly test with the prototype, later on tests with the PFM and follow.

% In zeitlicher Reihenfolgen auflisten, damit man sieht, zu welchem Zeitpunkt man mit welchen Teilen gearbeitet hat.
% In this section, all the different test (lab and simulations) are listed. As far as possible in their chronolocial order because between some lab tests there were simulations to improve the instrument before testing the redesigned instrument.

% Fügt ein PDF ein, nummeriert nach dem PDF normal weiter.
%\includepdf[pages=-]{Report_Thermofoil_UV_Masterarbeit.pdf}

% Section über Detector? haben wir ja nicht wirklich etwas gemacht. Diode durch 2 Widerstände ersetzt. -> Spannungsfestigkeit. Verlauf der Überschläge. Unterschiedliche Skizzen.

	In this section, the different tests are described to develop the NIM instrument. Different parts of the instrument were tested to improve the instrument.
%-----------------------------------------------------------------------------------
	\subsection{Ion-Mirror}
	
	The NIM prototype reflectron was exchanged through the flight like reflectron, which was tested. The NIM prototype reflectron consisted of 12 ring electrodes connected with each other with resistors in between them. On the first, 5th and 12th electrode, a voltage can be applied. With the different resistors, a linear voltage gradient in the reflectron is generated.\\ % Noch besser formulieren.
	The flight reflectron consists of a ceramic tube with two resistance spirals on its inner walls. There are three electrodes, where the voltage can be applied. The electrodes are connected via resistance spirals with each other. The two reflectrons can be seen in Fig.\,\ref{fig:ExpRefl}. This kind of reflectron was also used in the RTOF mass spectrometer which flied in ROSINA \cite{Diss_Scherer} and the in the NGMS \cite{Diss_Hofer}. \\ % Evt. noch etwas schöner und weiter ausführen. Strahlungsfestigkeit musste getestet werden. Report? Paper? Diss?
	Therefore, the two reflectrons are from the electrical point of view the same.\\ % Oder kommt das erst bei der Auswertung?
	
	\begin{figure}[h]
		\begin{subfigure}{0.5\textwidth}
			\centering
			\includegraphics[width = 0.95\textwidth]{Experiments/reflectron_Prototype1.jpg}
		\end{subfigure}
		\begin{subfigure}{0.5\textwidth}
			\centering
			\includegraphics[width = 0.85\textwidth]{Experiments/reflectron_flight.JPG}
		\end{subfigure}
		\caption{Left: Prototype reflectron with ringelectrodes. Right: Flight reflectron}
		\label{fig:ExpRefl}
	\end{figure}

	\subsubsection{Measurement Principle}\label{subsec:ReflecMeasPric}
		% Vortrag Messbedingungen zusammenschreiben
	
	\subsubsection{Discussion}\label{subsec:ReflecDissc}
		% Graphiken neu machen. Achsenskalierung, Spannungssets-Vergleich? Nur sagen, dass sie beinahe identisch sind.
	
	
	% Messbedingungen. Eigenes kurzes Unterkapitel machen
	% Auswertung, eigenes kurzes Unterkapitel

	\begin{comment}
	
	Exchange of the reflectron. Messdaten und Auswertung. Spannungssets vergleichen. Vortrag zusammenschreiben. Plots von dort nehmen und noch etwas besser beschreiben.
	(Plot-Axis, enlarge the exponent of the 10^11 e-/ns)
		
		
		
		
		
	\end{comment}
	

	
	
	
	\subsection{Prototype CASYMIR-D/-E}
	% CASYMIR-D? Erst wenn man intensitätsproblem mit ante chamber gelöst hat :/. CASYMIR-E-Kampagne.
	% Kurzfassung?
	
%-----------------------------------------------------------------------------------------
	\subsection{Simulations}
	
	During development, the mounting of the HV lenses was adapted. Simulations had to be done because as a result of the changed form of the lenses, the voltage set also changed. In this case, the voltage ranges increase by about blabla volts. These new higher ranges challenged the design of the supply electronics because the electronics has a limited amount of space. % Look up the details. Explanation with the electric fields?
	
	%\textcolor{red}{\textbf{Simulations of the adaption of the mounting of the electrodes has to come before the experiments with the PFM although the PFM structure is explained before.}}
	
	% Maybe an other arrangement of the Simulations chaper. See whicht simulations are performed how far any in which order they have to be to be comprehendible.
	
	% Evt. in einem eigenen Kapitel? Schauen, welchen Einfluss es dann jeweis auf die Hardware hatte. -> Elektronik setzt Spannungsranges, beschreiben, dass man da iterieren musste um das optimale Simulationsspannungsset zu finden mit den Grenzen, welche die Elektronik uns für die jeweiligen Elektroden gibt. -> vgl. mit den Messungen vom PFM.
	% Neusimulationen mit neuen Grenzen für die HV. -> die Resultate dieser Simulationen. Als Folgen davon wurden die Grenzen für die HV neu definiert. Schauen, wie man das am besten auseinander nimmt :/.

	% Change of the mounting of the IS lenses -> change in voltages

	% Pulsersimulationen -> Kriterien für Andy für das Pulserdesign. Ab wann wir einen Einbruch in der Massenauflösung haben. (Worddokument wo die einzelnen Bilder zusammengestellt sind?)
	
	% Filament repeller simulation tests. Noch Graphiken einmal einfügen. Die wichtigsten.
	% Um herauszufinden, wie wichtig die Position des Filaments ist.
	
	% Noch besser umschreiben. Die position des filaments entspricht nicht der erwarteten??? Was ist da schief gelaufen??? :(
	
	% Intensitätssimulation Countberechnung:
	% Man generiert 2000 e- auf dem Filament und zeichnet immer nach 1E-5 microsec auf, an welcher Position sich die Teilchen gerade befinden. Die Fkt. 'Plot_optVoltAndPos.m' zählt alle Positionen zusammen, welche sich in dem Zylindervolumen befinden. Die Grundfläche des Zylinders ist das Eintritts-Grid von der antechamber und die Höhe ist die Höhe der Entrance. Im th-Mode befinden sich nur in diesem Volumen neutrale Teilchen.
	\begin{figure}[h]
		\begin{subfigure}{0.53\textwidth}
			\centering
			\includegraphics[width= 0.95\textwidth]{Experiments/SimRepPosU.png}
		\end{subfigure}
		\begin{subfigure}{0.47\textwidth}
			\centering
			\includegraphics[width= 0.95\textwidth]{Experiments/SimRepPosImax.png}
		\end{subfigure}
		\caption{Left: The filament repeller voltage to reach the maximum electron intensity over the volume of the neutral particles. Right: Electron intensity normed on the intensity at position 0.} % Noch Graphik wie Intensität bei den versch. Pos abfällt.
		\label{fig:ExpSimRep}
	\end{figure}

	\subsection{Filament decision}
	% Ranking criterion -> Explanation, see electronics book. constant power of one filament -> resistance variies over time. Vgl. Diss Rico.

%-------------------------------------------------------------------------------------
	\subsection{Pulser}
	% Simulations with different rise times and their influence on the mass spectrum.
	% Test with lab pulser, wavelab pulser -> compare their properties and their performance.
	
	% Pulsertests, Messungen, Simulationsresultate
	% Two different pulsers properties, pulse shape. Tests with different gases. Massresolution, Intensity relation?, SNR? Are these two properties correlated? Noch mit Peter anschauen. Ar sieht in allen 4 Fällen etwas komisch aus. Higher signal intensity -> higher SNR.
	% Achsenskalierung anschauen von Areagraphik. Wenn das geklärt ist, evt. in Matlab schreiben. Zusätzliches Skript für diese Graphiken, Stefans Vorlagen anschauen. Die signal intensity lässt sich nicht einfach in Druck umrechnen. Zu viele unbekannte Komponenten -> a.u. oder # e-/ns angeben.
	
	
	
	\subsection{Detector Tests}
	% Plot off the gain curves of the detector in its different states. flat, folded, folded in the wolfram copper shealding (for 1 detector. Not the flight detector). If there should be any time, redo the errorcalculation correctly (error from the calculation of the area under the peak = charge and of the conversion factor when using a different discriminator level).
	% Measuring settings. Turn the drift voltage up to -2.5 kV and then slowly increase the anode voltage to the value you want to measure. The gain is calculated with the software by doing a Simpson 3/8 integration of the peak = Q. (Look in the lab book for the proper calculation)
	% Maybe there is time to redo the tests with the real PFM detector (XD wär schön. Leider nein). (Die Resultate sind nicht so vollständig wie die von der einen anderen Messung. Messung mit einem Zwillingsdetektor und von diesem auf den Flugdetektor schliessen. -> evt. schöne Graphik).
	
	
%----------------------------------------------------------------------------------------
	\subsection{Ionoptics}
	\subsubsection{Voltage Optimisation}
	Two types of electrical lenses. positive and negative voltage lenses. positive and negative voltage lenses have the same effect. In negative voltage lenses, the particles fly faster = shorter time-of-flight. This results in a better mass resolution.\\
	Aim in the lab is to get two different voltage sets. One for positive voltages to not stress the equipment and one with negative voltage lenses to reach the maximal performance of the instrument. -> Tests showed no significant better mass resolution. A more detailed data analysis has to be made. % data in folder: \\titania\UserHomes\foehn\My Documents\PhD\Messungen\PFM\CASYMIR-C\nMode_voltageSet_Comparison_200430 
	
	
	
	
	
	
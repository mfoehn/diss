% !TEX root = arbeit.tex
\section{Summary and Outlook}
	
	At the beginning of a space mission, there stands always a main scientific question. In case of JUICE its to investigate Jupiter and its icy moons as a planetary system potentially harbouring life. To answer that question, the question is divided into much more concrete questions out of which specific requirements for instruments participating in the space mission result. For the specific case of the NIM instrument, these requirements are the needed mass resolution, the signal-to-noise ratio and the mass range NIM has to detect to determine the composition of the icy moons' exosphere with enough accuracy to set further constraints for exosphere modellers to understand the formation and evolution processes involved to form the icy moons.\\
	The task of a scientist designing and building such an instrument is to identify the needed technical performance of the subsystems of the specific instrument based on the scientific requirements on the instrument performance mentioned above. Examples of such technical requirements are the range and stability of needed voltages to operate the instrument, switching times or gain. Other requirements arise from operations, e.g., the flyby trajectory and orientation of the spacecraft to the subject, which has to be investigated during the mission.\\
	In the following, two examples are shown on which the iteration processes of improving the instrument is shown. On the example of the high-voltage pulse generator it is shown, how important it is to push the operating electronics to their limit to reach higher performance with the new instruments leading to better data and an improvement of our knowledge. The second example focuses on the flyby trajectory and the restrictions arising for the instrument design from the mission itself.\\
	
	Two of the most important parameters of the high-voltage pulse generator used to extract the ions from NIM's ionisation region are the fall time and the bias voltage. The fall time has to be as short as possible to ensure that all ions get the same amount of energy. Otherwise, the mass resolution of especially low mass ions suffers (see Chap.~\ref{chap:massRes}). Typical fall times are in the range of a few ns and theory shows that this is a very critical parameter. Two different flight designs of the high voltage pulse generator were tested during the thesis using the prototype ion-optical system. The tests showed a clear preference to the one with the shorter fall time in agreement with the theory (see \cite{Lasi_IEEE2020}). Another important parameter is the bias voltage, which is the voltage applied on the extraction grid when no high voltage pulse is applied. As shown in Chap.~\ref{chap:FSCalib} and \cite{Foehn2021}, the PFM and FS ion-optical systems have both a good ion storage capability when operated with laboratory electronics. To achieve that it is necessary to have a stable potential well in the ionisation region during the time when no extraction pulse is applied on the extraction grid. This requires that the voltages applied on the electrodes in the ionisation region do not fluctuate more than 100~mV. Laboratory tests showed that the most critical ones with respect to ion storage are the bias voltage applied on the extraction grid and the voltage applied on the grid opposite of the extraction grid. The flight pulse generator shows less good performance with regards to that parameter than the laboratory electronics (see Chap.~\ref{chap:ExpPulser}). Due to other restrictions like the necessary radiation hardness of the electrical components limiting the number of options for certain components, this design was the best with the available resources.\\
	Other requirements appear from the trajectory of the spacecraft. JUICE contains several different instruments. The orientation of the spacecraft depends on the instruments to enable them an optimal FoV on their target. For a spacecraft having solar panels for power generation, it is important for the solar panels to be oriented perpendicular to the Sun whenever possible to optimize power generation especially for missions with targets such far away from the Sun as the mission JUICE with target Jupiter, which introduces constraints on the spacecraft orientation. The second priority for the spacecraft orientation is that the camera has the best view of the observation target on the moons' surface.\\
	For the NIM instrument in particular that meant to have a design with a big FoV to have a certain flexibility regarding the gas inflow direction into the instrument. NIM has an open-source and a closed-source entrance to measure neutral particles and ions directly and to thermalise neutral particles with the closed source antechamber to make advantage of the density enhancement behaviour to amplify the signal with that mode. Simulations of the trajectory of the spacecraft revealed that two entrance holes on the antechamber were necessary to be able to measure during the different flybys even though a second entrance hole leads to a lower performance of the closed-source antechamber (Chap.~\ref{subsubsec:Densenhan} and \ref{subsubsec:Calfly}).\\
	
	The NIM PFM was successfully delivered to the JUICE spacecraft in December 2020 and at the current state, the NIM FS model waits until the JUICE spacecraft started its journey to Jupiter in September 2022. Until then, the FS waits as it is the spare model for the case that something happens to the PFM on the spacecraft until launch. After the start of the spacecraft, the NIM FS has to be properly calibrated with the actual flight electronics. Most results presented in this thesis were conducted with laboratory electronics attached to the two ion-optical systems because there was only very little time to test the whole system. The NIM ion-optical system was first qualified as a separate unit and now there follows the calibration of the whole NIM instrument.\\
	In addition, the flight software is still under development and has to be tested with the full system. The optimiser to optimised the voltage sets during the flight is still under development. As soon as it is available for tests, it has to be tested and the target function used to improve the voltage sets has to be adapted for NIM. First results of the FS ion-optical system operated with flight electronics revealed that there lies a lot of potential in the postprocessing of the data especially in regards to filtering. Therefore a proper filter has to be written to improve the SNR of the final spectra.\\
	To have a proper calibration facility for the NIM instrument and future TOF instruments, the SATANS test facility (Supersonic cATion and ANion Source) is under development to generate neutral particle and ion beams with velocities from 1--15~km/s. The CASYMIR test facility is only able to generate neutral particle beams with velocities up to 4~km/s but the NIM instrument has to detect neutral particles and ions with velocities up to 8~km/s. Therefore, SATANS was developed to have a facility covering a higher velocity range for neutral particles and also for ions. Such a calibration facility in necessary to calibrate the FS instrument and to replicate the data recorded with the actual flight instrument on the satellite to understand the measured data. Therefore, a test setup is required able to reproduce the environment is space as close as possible. First tests performed with the NIM prototype on SATANS showed promising results \cite{SATANS_Meyer2018} but there is still a lot of potential in improving the performance and stability of SATANS. At the current state, SATANS is able to produce an ion beam with velocities between 1~--~15~km/s. As soon as the stability of the ion beam is improved, the system will be upgraded with a neutraliser to also produce a neutral particle beam in this energy range.
	
	
	
	
	
	
	
	
	
	
	
	
	
	
	
	
	
	
	
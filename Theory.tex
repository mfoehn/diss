% !TEX root = arbeit.tex
\section{Theory}

	\subsection{Requirements}
	% Mass, performance, Power consumption. (See latest references. Part of the introduction?)
	
	\subsection{Basic Theory about a TOF masspectrometry} % Allgemeine Theorie
	
	\subsubsection{Principle} % Noch Referenzen auf Vorlesung einfügen. Evt. noch Bild eifügen von Ionen an versch. Startpositionen.
	This chapter explains the function of a TOF instruments. A TOF mass spectrometer consists of, an ion-source, a mass analyser and a detector. The mass analyser has an ion-mirror which increases the flight distance of the ions by keeping the instrument on a certain length. A longer flight distance increases the mass resolution of the instrument.\\% Mention that later.
	In the ion source, the ions are produced. An electric field in the source generated such that ions get trapped in it. Then ions get accelerated by applying a high voltage pulse on the extraction grid. If the pulse width is long enough that all ions leave the source during the time the pulse is applied, the ions get all the same amount of energy W
	\begin{equation}
		W = \int_{0}^{s_0}q_0 E_s ds = q_0 U_0
	\end{equation}
	With $s_0$ the distance the ions get accelerated, in our case 1\si{\milli\metre} which is half the height of the ion source, $q_0$ the elementary charge, $E_s$ the applied electric field and $U_0$ the voltage applied on the extraction grid. This energy is equal to the kinetic energy the ions have after leaving the ion-source
	\begin{equation}
		q_0 U_0 = \frac{1}{2}m v^2
	\end{equation}
	With $m$ the mass of the particle and $v$ the particle velocity. Rearranging this formula we get
	\begin{equation}
		\frac{m}{q_0} = 2 U_0\frac{t^2}{s^2}
		\label{eq:m/q}
	\end{equation}
	With $t$ the time of flight and $s$ the flight distance. Therefore, $m/q_0$ is proportional to $t^2$.\\
	Ions starting at different positions also get a different amount of energy. Ions starting closer to the extraction grid will get less energy than ions further away from the grid. At a certain point after the source, the ions with higher energy have overtaken the slower ions. This point is at around $2\cdot s_0$ which corresponds to a very short flight distance. To shift this focal point onto to the detector, additional fields after the ion-source are applied.\\
	The ions have different thermal energies. Therefore ions of the same mass and starting at the same distance from the extraction grid will not have all the same energy when they leave the ion source. This energy spread leads to a difference in their velocity and to different flight times. This energy spread can be partially compensated by an ion-mirror also referred to as reflectron (Fig.\,\ref{fig:NIMSketch}). Ions with a higher energy will penetrate deeper into the ion-mirror and have a bigger flight path than ions with less energy. Therefore, the ion-mirror is able to compensate the different start energies of the ions. In the worst case scenario we have one particle flying toward the extraction grid and the other particle flying with the same velocity in the opposite direction. The second particle gets decelerated and has to turn in the source. When it reached its initial position, it has the same amount of energy as the first particle. But it will always be behind the first particle by a constant time it needed to turn around and reach its initial position. To minimize this effect, one has to minimise the distance of the ionisation region or increase the voltage of the HV pulse. But a smaller ion source means less ions and therefore less signal. Increasing the HV pulse results in bigger electronic noise at the start of the spectrum. The biggest impact is there on the light particles such as $H$ or $H_{2}$. Therefore, one has to make a trade-off. % Evt. noch Effekte etwas besser ausführen.

	\subsubsection{Mass Calibration}
	In this section we discuss the calibration of the mass axis. According to Eq.\,\eqref{eq:m/q} the mass/\,charge is proportional to the square of time. By rearranging Eq.\,\eqref{eq:m/q}
	\begin{equation}
		m = 2 q_0 U_0 \frac{t^2}{s^2}
		\label{eq:mass_Calib_pre}
	\end{equation}
	Take together all parameter which remain constant to one single constant:
	\begin{equation}
		C = \frac{2 q_0 U_0 }{s^2} 
	\end{equation}
	and considering the time scale has a constant offset $t_0$, equation \eqref{eq:mass_Calib_pre} results in
	\begin{equation}
		m = C(t-t_0)^2
		\label{eq:mass_Calib}
	\end{equation}
	In this equation there are two free parameters, $C$ and $t_0$. To calibrate a mass spectrum at least two species appearing in the spectrum have to be known to solve this equation for the two parameters.
	
	
	\subsubsection{Mass resolution}
	According to bla % Ref.
	the mass resolution is:
	\begin{equation}
		\frac{m}{\Delta m} = \frac{t}{2 \Delta t}
	\end{equation}
	
	\subsubsection{Sensitivity}
	
	% Then continue with the calculation of the mass resolution and the calibration of the mass scale (evt. Reference further explanation how the software it acctually does (keep it short, only extend it if further nessecary)).
	
	% SNR, Sensitivity
	
	% Uncertainty cause by the 'Umkehrzeit'. Distance between 2 and 3 is always the same and depends on Uth
	
	% Include a schematics of the ion source with the different states/ effects.
	% include a picture of a sample source.
	
	\subsection{Ion Optical Design, NIM specific elements} % ion source efficienies, reflectron double focusing, detector reflection of signal, signal matching. Density enhancement
	A time of flight mass spectrometer consists of, an ion-source, a mass analyser and a detector.\\
	
	\begin{figure}[htb] % Noch schauen, ob das noch verschoben wird.
		\centering
		\includegraphics[width= 10cm]{Bilder/NIM_Sketch.png} % Bei Bild noch schauen, ob die Ränder drauf sind. Bei Zeiten noch Bild anpassen.
		\caption{Schematics of the NIM mass spectrometer. Adapted from \cite{Diss_Meyer}.}
		\label{fig:NIMSketch}
	\end{figure}

	%First overview and then go into the details.
	The NIM instrument is able to measure neutrals and ions. Neutral particles get ionised by electron ionisation. A filament is heated up until it emits electron. Ions enter the ion source directly.  % Noch genaue Formulierung nachschauen.
	All ions then get accelerated to the same energy and fly through the mass analyser. Light particles fly faster through the spectrometer than heavier ones. The different particle species arrive at different times at the detector. To enlarge the flight distance, an ion-mirror, which reflects the ions and leads them back to the detector. The used detector is a MCP detector. % Explain a little bit in more detail.
	
		\subsubsection{Ion-source}
		% More detailed explanation about the NIM ion source, function of the different specific electrodes. Or explain it later in detail when discussing the simulations in detail.
		To calculate the number of ions produced in the ion source we use:
		\begin{equation}
		I_{ion} = \beta\cdot Q_{ion}\cdot L\cdot n\cdot I_{em}
		\end{equation}
		With $\beta$ the extraction efficiency which is 1, % Noch schauen auf welchen Wert wir diese setzen wollen. 1= sehr gute Quelle, 0.01 = 1mus/100mus = Pulslänge Pulser/ Länge 1 Zyklus. Noch diskutieren, welche Werte beta haben kann oder einfach etwas setzen? Da die ersten Messresultate gut sind, würde ich eher auf 1 setzen.
		$L$ as the effective ionising path in our case 4~\si{\milli\metre}, $n$ the particle density, $I_{em}$ the electron emission current from the filament and $Q_{ion}$ the ionising cross section. The cross sections of species used in our calibration can be found in table % Ref. auf Tabelle und nur auf Stefans Diss verweisen oder die 4 Originalpaper zusammen suchen.
		
		
		
		% Reference: Bieler Diss 2012, Wurz 2011, Scherer 1999, Meyer 2013 Data Analysis, Wells 2011
		% Ionisation efficiency
		% Description of how it works with the energies. Electric -> kinetic. Pulser. E = 1/2 mv^2 = qU
		% Time focusing?
		% Mass calibration t/dt -> m/dm.
		% SNR definition. Picture?
		% Sensitivity estimation. Nessesary? If so, part of SNR discussion.
		% MCP detector. Gain calculation. How the detecotr works. Didn't do anything for further developement...
		% Antchamber. Explanation of closed and open source. Field of view. Densitiy enhancement.
	
	
\begin{comment}
	
	Explain how a TOF works. Source, reflectron, detector.
	Explanation of the antechamber at the end after explaining the different parts.
	
	Detection efficiency Ionsource, MCP? -> y-Achse
	Mass Analyser, mass spectrometer. Most important formulas. dm/m = dt/t...
	Sensitivity
	Density enhancement -> explanation about closed and open source
	Sketch of the instrument
	Requirements: Power, mass, mass resolution,... (At the end or at the beginning. Introduction)
	
	
	Ionisationseffizienz/ Ionenproduktion der wichtigsten Gase. Detektionseffizienz hängt auch von Effizienz der MCPs ab... Erklären weshalb man die Achsenskalieruing braucht (entweder Counts oder die angepasste a.u. bei der Fläche und Counts/s für das Spektrum)
	
\end{comment}
		
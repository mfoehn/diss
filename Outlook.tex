% !TEX root = arbeit.tex
\section{Outlook}
	
	\notes{Proofread} The NIM PFM was successfully delivered to the JUICE spacecraft in December 2020 and at the current state, the NIM FS model waits until the JUICE spacecraft started its journey to Jupiter in September 2022. Until then, the FS waits as it is the spare model for the case that something happens to the PFM on the spacecraft until launch. After the start of the spacecraft, the NIM FS has to be properly calibrated with the actual flight electronics. Most results presented in this thesis were conducted with laboratory electronics attached to the two sensors because there was only very little time to test the whole system. The NIM ion-optical system was first qualified as a separate unit and now there follows the calibration of the whole NIM instrument.\\
	In addition, the flight software is still under development and has to be tested with the full system. The optimizer to optimized the voltage sets during the flight is still under development. As soon as it is available for tests, it has to be tested and the target function used to improve the voltage sets has to be adapted for NIM. First results of the FS sensor operated with flight electronics revealed that there lies a lot of potential in the postprocessing of the data especially in regards to filtering. Therefore a proper filter has to be written to improve the SNR of the final spectra.\\
	To have a proper calibration facility for the NIM instrument and future TOF instruments, the SATANS test facility is under development to generate neutral particle and ion beams with velocities from 1--15~km/s. The CASYMIR test facility is limited to neutral particle velocities up to 4~km/s but to test with NIM the full velocity range, SATANS was developed to has a facility covering a higher velocity range for neutral particles and also for ions. Such a calibration facility in necessary to calibrate the FS instrument and to replicate the data recorded with the actual flight instrument on the satellite to understand what we measured out there. Therefore, a test setup is required able to reproduce the environment is space as close as possible. First tests performed with the NIM prototype on SATANS showed promising results \cite{SATANS_Meyer2018} but there is still a lot potential in improving the performance and stability of SATANS. At the current state, SATANS is able to produce an ion beam with velocities between 1--15~km/s. As soon as the stability of the ion beam is improved, the system will be upgraded with a neutraliser to also produce a neutral particle beam in this energy range.
	
	
	
	